% Retoca las líneas marcadas con TODO según las necesidades

\documentclass[oneside,a4paper,12pt]{book} % TODO: cambia "oneside" por "twoside" a la hora de imprimirlo

\usepackage[spanish]{babel}
\usepackage[utf8]{inputenc}
\usepackage{geometry}
\usepackage{makeidx}
\usepackage{url}
\usepackage{xurl}
\usepackage{graphicx}
\usepackage{color}
\usepackage{caption}
\usepackage{acronym}
\usepackage{hyphenat}
\usepackage{a4wide}
\usepackage[normalsize]{subfigure}
\usepackage{float}
\usepackage{titlesec}
\usepackage{multirow}
\usepackage{tabularx}
\usepackage{array}
\usepackage[Lenny]{fncychap}
\usepackage{listings} % para poder hacer uso de "listings" propios (p.ej. códigos)
\usepackage{eurosym} % para poder usar el símbolo del euro con \euro {xx}
\usepackage{hyperref} % TODO: añade la opción hidelinks para imprimirlo (los enlaces no aparecerán resaltados)

% Para que no parta las palabras
%pretolerance=10000

\newcommand{\bigrule}{\titlerule[0.5mm]} \titleformat{\chapter}[display] % cambiamos el formato de los capítulos
{\bfseries\Huge} % por defecto se usaron caracteres de tamaño huge en negrita
{% contenido de la etiqueta 
\titlerule % línea horizontal 
\filright % texto alineado a la derecha 
\Large\chaptertitlename\ % capítulo e índice en tamaño large
\Large % en lugar de 
\Huge \Large\thechapter} 
{0mm} % espacio mínimo entre etiqueta y cuerpo
{\filright} % texto del cuerpo alineado a la derecha
[\vspace{0.5mm} \bigrule] % después del cuerpo, dejar espacio vertical y trazar línea horizontal gruesa
\geometry{a4paper, left=3.5cm, right=2cm, top=3cm, bottom=2cm, headsep=1.5cm}

% Estilos para ilustrar códigos:
\definecolor{code_green}{rgb}{0,0.6,0}
\definecolor{code_gray}{rgb}{0.5,0.5,0.5}
\definecolor{code_mauve}{rgb}{0.58,0,0.82}

\lstset{frame=tb,
  language=C,
  aboveskip=3mm,
  belowskip=3mm,
  showstringspaces=false,
  columns=flexible,
  basicstyle={\small\ttfamily},
  numbers=none,
  numberstyle=\tiny\color{code_gray},
  keywordstyle=\color{blue},
  commentstyle=\color{code_green},
  stringstyle=\color{code_mauve},
  breaklines=true,
  breakatwhitespace=true,
  tabsize=3
}

\lstset{frame=tb,
  language=C++,
  aboveskip=3mm,
  belowskip=3mm,
  showstringspaces=false,
  columns=flexible,
  basicstyle={\small\ttfamily},
  numbers=none,
  numberstyle=\tiny\color{code_gray},
  keywordstyle=\color{blue},
  commentstyle=\color{code_green},
  stringstyle=\color{code_mauve},
  breaklines=true,
  breakatwhitespace=true,
  tabsize=3
}

\lstset{frame=tb,
  language=Python,
  aboveskip=3mm,
  belowskip=3mm,
  showstringspaces=false,
  columns=flexible,
  basicstyle={\small\ttfamily},
  numbers=none,
  numberstyle=\tiny\color{code_gray},
  keywordstyle=\color{blue},
  commentstyle=\color{code_green},
  stringstyle=\color{code_mauve},
  breaklines=true,
  breakatwhitespace=true,
  tabsize=3
}

% Definición de mis propios tipos: Códigos, Ecuaciones y Tablas
\DeclareCaptionType{code}[Código][Listado de códigos]
\DeclareCaptionType{myequation}[Ecuación][Listado de ecuaciones]

% TODO: especifica las reglas de separación que consideres. Algunos ejemplos:
\hyphenation{fuer-tes}
\hyphenation{mul-ti-ca-pa}
\hyphenation{res-pues-ta}
\hyphenation{di-fe-ren-tes}
\hyphenation{de-sa-rro-lla-dos}
\hyphenation{re-pre-sen-tan-do}

