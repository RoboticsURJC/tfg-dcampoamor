\chapter{Objetivos}
\label{cap:capitulo3}
\setcounter{footnote}{12}
 
Una vez presentado el contexto general en el cual se enmarca el presente trabajo de fin de grado, en este capítulo se describen los objetivos y requisitos de este, así como la metodología y el plan de trabajo llevados a cabo.

\section{Descripción del problema}
\label{sec:descripcion}

La necesidad de implementar soluciones tecnológicas que automaticen y optimicen las tareas de recolección incrementando la eficiencia en la recolección, mejorando la calidad del producto y disminuyendo los costes asociados, surge debido a la situación actual de la agricultura en la que, uno de los mayores desafíos que enfrenta es la recolección de frutas y hortalizas, problema que deriva de la escasa mano de obra disponible y el proceso manual que esto conlleva.%, y de la posibilidad de que existan errores humanos en la identificación de los frutos para su recolección, pudiendo influenciar esto en la calidad del producto, especialmente en la recolección de frutos que requieren un manejo cuidadoso, como las fresas.\\

La solución propuesta en este trabajo busca ayudar a mejorar esta situación, proporcionando un robot de bajo coste y accesible a cualquier persona, que sirva para poder mejorar el proceso de reconocimiento por visión de la maduración de frutos, más concretamente fresas, para su posterior recolección. Por lo tanto, este proyecto pretende, como objetivo principal, utilizar un robot colaborativo que, gracias a su interfaz intuitiva sea accesible a cualquier persona y, junto con el sistema de detección elaborado con materiales de bajo coste, sea capaz de reconocer las fresas maduras de un sistema de cultivo agrícola vertical, para su posterior recolección por el brazo robótico, gracias a la comunicación establecida entre el sistema de visión y el robot.

Con el fin de alcanzar este objetivo principal, se han establecido los siguientes
subobjetivos:

\begin{enumerate}
  \item Investigar las soluciones actuales que cumplen con las características y objetivos establecidos.
  \item Seleccionar la técnica de inteligencia artificial de reconocimiento de frutas y seleccionar los componentes hardware necesarios para desarrollar el sistema de visión de bajo coste más eficiente.
  \item Optimizar la técnica escogida y adaptarla de tal manera que sea capaz de funcionar en nuestra plataforma. Al ser una técnica basada en Machine Learning, se deberá crear un dataset con imágenes de fresas y, por lo tanto, hacer un correcto tratamiento de los datos para conseguir un resultado preciso en el posterior entrenamiento.
  \item Realizar el entrenamiento con varios algoritmos de Machine Learning de
clasificación. Estudiar el rendimiento y precisión de cada uno de ellos a través de pruebas con el sistema de visión y fresas reales.
  \item Seleccionar el protocolo de comunicación entre el sistema de visión y el robot y llevar a cabo pruebas; tanto simuladas, a través del simulador que facilita el fabricante del robot, como reales, para establecer esta comunicación.
  \item Dar soporte software al robot mediante un %Programación tanto del robot como del archivo en Python que posee el código del 
sistema de reconocimiento de fresas, que guarde las posiciones y la distancia de estas a la posición de la cámara, para su posterior envío al brazo robótico.
  \item Realizar pruebas de la aplicación final, tanto en entornos simulados como reales.
\end{enumerate} 
 
\section{Plan de trabajo}
\label{sec:plantrabajo}
El desarrollo y seguimiento que el proyecto ha seguido es una planificación en base a reuniones semanales con el tutor, en las cuales se revisaron los avances, se fijaron nuevos objetivos y se discutieron y propusieron posibles mejoras, mientras que el trabajo se organizó en varias fases clave: 
\begin{enumerate}
  \item \textit{Investigación inicial:} En esta fase, se investigó el estado del arte relacionado con sistemas de visión artificial y técnicas de reconocimiento de objetos, especialmente aplicadas a la maduración de frutas y hortalizas, y utilizando para ello artículos científicos, capítulos de libros y proyectos previos. 
  \item \textit{Diseño y desarrollo del sistema de visión artificial:} Esta fase se centró en el diseño y la implementación del sistema de visión artificial, abarcando tanto el desarrollo del software como la integración del hardware, e incluyendo la calibración y obtención de los parámetros intrínsecos a la cámara y las diversas pruebas realizadas con distintos sistemas y códigos, hasta seleccionar el \textit{software} funcional con el que se llevó a cabo el proyecto finalmente.
  \item \textit{Pruebas en entorno simulado:} Durante esta fase se realizaron múltiples pruebas y ajustes para optimizar el funcionamiento del sistema y comprobar su funcionamiento en diferentes escenarios, simulando de manera separada la programación del robot, para el que se utilizó un simulador en una máquina virtual, y la detección y funcionamiento del sistema de visión, cuyos algoritmos se afinaron para mejorar la precisión en la detección y se ajustaron los parámetros relacionados con la cámara en los códigos para poder obtener las coordenadas y distancia real de las detecciones respecto a la cámara y poder transmitírselas al brazo robótico. Finalmente, también se llevaron a cabo pruebas de comunicación entre el sistema de visión y el robot, poniendo a prueba su programación, para que este alcanzase el punto de la detección.
  \item \textit{Pruebas en entorno real:} Una vez desarrollado el prototipo inicial, el sistema completo fue sometido a pruebas en un entorno real de lo que sería la aplicación final. 
  \item \textit{Escritura de la memoria:} Con el sistema ya afinado y probado, se procedió a la redacción de la memoria del proyecto. En esta etapa, se documentó detalladamente todo el proceso seguido, desde la investigación inicial hasta los resultados finales obtenidos durante las pruebas reales. 
\end{enumerate}


Todo el contenido del proyecto se puede encontrar en un repositorio público de GitHub\footnote{\url{https://github.com/RoboticsURJC/tfg-dcampoamor}}, en cuya wiki\footnote{\url{https://github.com/RoboticsURJC/tfg-dcampoamor/wiki}} se puede ver el desarrollo del trabajo en semanas a lo largo de los meses, durante el trascurso del proyecto. Las requisitos necesarios para la consecución de los objetivos planteados, las competencias desarrolladas y la metodología empleada pueden encontrarse descritos en el Anexo \ref{cap:capitulo7}.
%Después de haber revisado los objetivos, requisitos, competencias, metodología y el plan de trabajo implementado para la realización de este proyecto, en el siguiente capítulo se abordarán las plataformas de desarrollo empleadas.

