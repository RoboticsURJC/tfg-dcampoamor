\cleardoublepage

\chapter*{Acrónimos\markboth{Acrónimos}{Acrónimos}}

% Añade a continuación los acrónimos que uses en el documento. Algunos ejemplos:
\begin{acronym}
        \acro{ABB}{\emph{Asea Brown Boveri}}
	\acro{AER}{\emph{Asociación Española de Robótica}}
	\acro{AERO}{\emph{Autonomous Exploration Rover}}
	\acro{AI}{\emph{Artificial Intelligence}}
	\acro{ANN}{\emph{Artificial Neural Network}}
	\acro{API}{\emph{Application Programming Interface}}
	\acro{DLR}{\emph{Instituto de Robótica y Mecatrónica del Centro Aeroespacial Alemán}}
	\acro{EKF}{\emph{Extended Kalman Filter}}
	\acro{FOA}{\emph{Focus of Attention}}
	\acro{GA}{\emph{Genetic Algorithm}}
	\acro{GPIO}{\emph{General Purpose Input/Output}}
	\acro{GPS}{\emph{Global Positioning System}}
	\acro{HCI}{\emph{Human-Computer Interaction}}
	\acro{HRI}{\emph{Human-Robot Interaction}}
	\acro{IA}{\emph{Inteligencia Artificial}}
	\acro{IBM}{\emph{International Business Machines}}
	\acro{IFR}{\emph{International Federation of Robots}}
	\acro{ISO}{\emph{Internacional Organization for Standardization}}
	\acro{LWR}{\emph{Lightweight Robot}}
	\acro{OSRF}{\emph{Open Source Robotics Foundation}}
	\acro{ROS}{\emph{Robot Operating System}}
	\acro{SAIL}{\emph{Stanford Artificial Intelligence Laboratory}}
	\acro{SRI}{\emph{Stanford Research Institute}}
	\acro{TC}{\emph{Technical Committee}}
	\acro{UR}{\emph{Universal Robots}}
\end{acronym}
