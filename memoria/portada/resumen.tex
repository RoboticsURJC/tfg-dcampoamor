\cleardoublepage

\chapter*{Resumen\markboth{Resumen}{Resumen}}

La robótica y la visión artifical han revolucionado numerosos sectores, incluida la agricultura, en la que, a pesar de los avances tecnológicos, la recolección manual de las frutas y verduras sigue siendo un proceso laborioso, exigente y sujeto a tareas repetitivas susceptibles de derivar en errores humanos.

Uno de los mayores desafíos en este campo es la recolección de frutas pequeñas y delicadas, como lo son en particular las fresas dada su gran variabilidad en tamaño, forma y grado de maduración; ya que requieren gran precisión y un alto consumo de tiempo y esfuerzo físico por quienes lo realizan. Es por esto que la automatización de su recolección se ha convertido en una alternativa para poder mejorar y optimizar su eficiencia, reduciendo la dependencia de la mano de obra humana mediante el uso de la robótica y la inteligencia y visión artificial para identificar, seleccionar y recoger los frutos en el momento óptimo.

El presente trabajo pretende solucionar este problema mediante el desarrollo de un sistema de visión artificial para detectar el estado de maduración de las fresas y facilitar su recolección de forma automatizada, siempre y cuando el estado de maduración de la fresa sea el adecuado, con un brazo robótico y utilizando el modelo YOLOv3 en tiempo real. Mediante el procesamiento de las imágenes capturadas por una cámara, el sistema identifica la posición y calcula la distancia de cada fresa con respecto a la cámara para poder transmitir esta información a un brazo robótico de Universal Robots a través del protocolo XML-RPC, permitiendo que el robot ejecute esta recolección de forma autónoma y precisa.

Los experimentos realizados han demostrado que el sistema puede identificar fresas maduras con alta precisión en distintas condiciones de iluminación. Además, la integración con el brazo robótico ha permitido validar la eficacia del sistema en la recolección autónoma, logrando resultados satisfactorios en términos de exactitud. Estos avances confirman la viabilidad de la propuesta y sientan las bases para futuras mejoras en rendimiento, velocidad y adaptabilidad a otros cultivos.


