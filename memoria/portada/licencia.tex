% Este diseño se corresponde con la licencia CC-BY-NC-SA.
% Por supuesto, puedes poner la licencia que mejor se adapte al propósito de tu trabajo.
% Recuerda que, si no se especifica ninguna licencia, esta -como cualquier creación artística- pasaría a estar licenciada con todos los derechos reservados (copyright).


\thispagestyle{empty}
\begin{center}
    \includegraphics[width=0.45\textwidth]{figs/logo_urjc.jpg}\\[1cm]

    {\large{\textbf{Grado en Ingeniería de Tecnologías Industriales}}}\\[1cm]
    {\large{\textbf{Trabajo de Fin de Grado}}}\\[1cm]
\end{center}


El presente trabajo, titulado \textit{\textbf{SISTEMA DE RECONOCIMIENTO POR VISIÓN DE MADURACIÓN DE FRUTOS PARA SU RECOLECCIÓN CON UN BRAZO ROBÓTICO}}, constituye la memoria correspondiente a la asignatura Trabajo de Fin de Grado que presenta D./D\textsuperscript{a}. \textit{\textbf{DAVID CAMPOAMOR MEDRANO}} como parte de su formación para aspirar al Título de Graduado/a en Ingeniería de Tecnologías Industriales. Este trabajo ha sido realizado en \textit{\textbf{ESCUELA DE INGENIERÍA DE FUENLABRADA}} en el \textit{\textbf{DEPARTAMENTO DE TEORÍA DE LA SEÑAL Y COMUNICACIONES Y SISTEMAS TELEMÁTICOS Y COMPUTACIÓN}} bajo la dirección de \textit{\textbf{JULIO VEGA PÉREZ}}.\\[3cm]

\begin{flushright}
Móstoles, 12 de mayo de 2025
\end{flushright}

\cleardoublepage

\begin{figure}
 \ \ \ \ \includegraphics[width=0.25\linewidth]{figs/by-nc-sa.png}
 \label{fig:cc} 
 \end{figure}

\

\

\

\noindent
Este trabajo se distribuye bajo los términos de la licencia internacional \href{http://creativecommons.org/licenses/by-nc-sa/4.0/}{CC BY-NC-SA International License} (Creative Commons AttributionNonCommercial-ShareAlike 4.0). Usted es libre de \textit{(a) compartir}: copiar y redistribuir el material en cualquier medio o formato; y \textit{(b) adaptar}: remezclar, transformar y crear a partir del material. El licenciador no puede revocar estas libertades mientras cumpla con los términos de la licencia:

\begin{itemize}
\item \textit{Atribución}. Usted debe dar crédito de manera adecuada, brindar un enlace a la licencia, e indicar si se han realizado cambios. Puede hacerlo en cualquier forma razonable, pero no de forma tal que sugiera que usted o su uso tienen el apoyo de la licenciante.
\item \textit{No comercial}. Usted no puede hacer uso del material con propósitos comerciales.
\item \textit{Compartir igual}. Si remezcla, transforma o crea a partir del material, debe distribuir su contribución bajo la la misma licencia del original.
\end{itemize}

\begin{flushright}
		\vspace{7.0 cm}
		\emph{Documento de} \textbf{David Campoamor Medrano}. % TODO: pon aquí tu nombre cuando hagas el documento
\end{flushright}

