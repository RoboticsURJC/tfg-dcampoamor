\cleardoublepage

\chapter*{Abstract\markboth{Abstract}{Abstract}}

Artificial intelligence and robotics have revolutionised numerous sectors, including agriculture, where, despite technological advances, the manual harvesting of fruits and vegetables remains a labour-intensive, demanding process, prone to repetitive tasks and human error.

One of the greatest challenges in this field is the harvesting of small and delicate fruits, such as strawberries, which exhibit high variability in size, shape, and ripeness level. These fruits require great precision and significant physical effort and time from those who harvest them. For this reason, the automation of harvesting has become an alternative to enhance and optimise efficiency, reducing dependence on human labour through the use of robotics and artificial intelligence and vision to identify, select, and harvest the fruits at the optimal moment.

This project aims to address this problem by developing a computer vision system capable of detecting the ripeness stage of strawberries and facilitating their automated harvesting, provided that the fruit is at the appropriate stage. The system employs a robotic arm and uses the YOLOv3 model in real time. By processing images captured by a webcamera, the system identifies the position and calculates the distance of each strawberry from the camera in order to transmit this information to an Universal Robots robotic arm via XML-RPC protocol, allowing the robot to perform harvesting in an autonomous and precise manner.

The experiments conducted have demonstrated that the system can identify ripe strawberries with high accuracy under varying lighting conditions. Furthermore, integration with the robotic arm has validated the system’s effectiveness in autonomous harvesting, yielding satisfactory results in terms of precision. These advances confirm the feasibility of the proposed approach and lay the foundation for future improvements in yield, scalability, and adaptability to other crops.



