\cleardoublepage

\chapter*{Agradecimientos}

Nunca es tarea fácil agradecer a tantas personas el apoyo, la ayuda y los consejos que han contribuido en mi beneficio, tanto personal como académico, durante todos estos años.\\

En primer lugar, me gustaría dar las gracias tanto a la Universidad Rey Juan Carlos como a todos los profesores de los que he tenido el privilegio de ser alumno, por haber sido capaces de transmitir la dedicación, pasión, disciplina y el esfuerzo tan imprescindible como necesarios para la praxis de una profesión como lo es la de ingeniero, y más concretamente en mi caso, la de ingeniero industrial. \\
Quisiera expresar mi gratitud a mi tutor, Julio Vega, por guiarme, acompañarme y ayudarme durante estos meses de trabajo, para mi fue todo un honor saber que finalmente había aceptado dirigir este trabajo final de grado, y de este modo cerrar un bonito círculo que empezó con él como profesor mío de informática en el colegio, donde nos enseñó, entre otras muchas cosas, que más allá de los editores de texto convencionales, existen otros sitemas para la preparación de documentos, por esto, este trabajo también es en parte suya, ya que tanto estas líneas como el resto del documento están basados en sus enseñanzas.\\

Asímismo, me gustaría agradecer a Robotplus, por cumplimentar mi formación académica y darme mi primera oportunidad laboral en el ámbito industrial, y más concretamente, a mis compañeros del departamento de servicio técnico y a los compañeros del departamento de I+D+i, ya que gracias a ellos hoy por hoy he podido entender y experimentar más en profundidad muchos de los principios teóricos y de los problemas que únicamente conocía sobre el papel, pudiendo desarrrollarme de una manera más completa como profesional.\\

 Agradecer también a mis amigos y compañeros de clase, los \textit{Hijos de la Ingeniería} y David, por no haber dejado que me rindiera incluso en los peores momentos y con todo en contra, y por haber sido un gran apoyo tanto dentro como fuera de la universidad.
 A mis amigos del equipo de baloncesto en Alcorcón, en especial a Adri, por haber confiado siempre en que este momento llegaría,antes o desùés, y haber formado parte de este proceso del que desde antes de empezar la universidad ya formaba parte, al igual que mis amigos de Móstoles del colegio, el \textit{Cártel de La Manga}. Y sobre todo, gracias a Sandra, por ser para mí el claro ejemplo de que la dedicación y el trabajo duro merecen la pena, pero más allá de todo esto, por estar a mi lado día a día y ser mi compañera de vida, sin ella no habría podido soñar con finalmente leagar hasta aquí.\\

No querría concluir los agradecimientos sin hacer partícipe a toda mi familia, y en especial a mis padres y mi hermano, la paciencia que han tenido todo este tiempo conmigo, sobre todo en época de entregas y de exámenes, pero sobre todo y más importante, la confianza depositada en mí, que mediante palabras y gestos de apoyo incondicional han demostrado. Ha sido gracias a este amor y apoyo que solo la familia sabe darte cuando más lo necesitas, por lo que ha sido más fácil poder alcanzar esta meta. Gracias a mis tíos y a mis primos mayores, por hacer que me interesara en el mundo de las ciencias, y más concretamente en la ingeniería y la construcción, faceta en la que ya desde pequeño había fijado mi atención jugando con aquellos bloques fabricados en plástico ABS y de colorines; ya que sin duda, fue gracias a ellos por lo que terminé de decidir embarcarme, ya desde el colegio, en las materias que guardaban mayor similitud con estos aspectos antes que en otras, puesto que veía en ellos, una referencia a seguir. Pero sobre todo, gracias a mis abuelos y mis yayes, que como suele decirse, deberían ser eternos, si antes hablaba de referencias, sin duda ellos han sido el máximo exponente en esto, puesto que sin sus enseñanzas y consejos, y no solo en aspectos académicos, no podría haber llegado hasta aquí. Todos ellos siempre formarán parte de mí y estarán presentes en cada una de las tomas de decisiones importantes que tenga que llevar a cabo, en las desilusiones y en los malos ratos, pero también en la consecución de mis éxitos y mis logros, como es el caso, aunque algunos de ellos ya no se encuentren entre nosotros o no puedan recordarlo. Espero haber podido aprender y retener algo de la sabiduríaque me habéis mostrado y transmitido.\\ 

\begin{flushright}
		\emph{A todas aquellas personas que, con trabajo y esfuerzo,\\
terminan consiguiendo todo aquello que se proponen.}\\
		\par
		\vspace{1.0 cm}
		Madrid, xx de xxxxxx de 20xx\\ %\today
		\emph{David Campoamor Medrano}
\end{flushright}

\thispagestyle{empty}

